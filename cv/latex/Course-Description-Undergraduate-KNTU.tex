

% Document settings
\documentclass[12pt]{article}
\usepackage[margin=0.70in]{geometry}
\usepackage[pdftex]{graphicx}
\usepackage{multirow}
\usepackage{setspace}
\usepackage{enumitem}
\usepackage{tabularx}
\usepackage{float}
\pagestyle{plain}
\setlength\parindent{0pt}
\setlist[itemize]{leftmargin=5.5mm}
\title{Course Description}



\begin{document}



\maketitle






\begin{minipage}{\textwidth}
\begin{tabularx}{\textwidth}{|l|X|l|X|}
\hline
\textbf{Course Title}       &  Calculus 1 & \textbf{Course Code}       &  5712094 \\ \hline
\textbf{Credit Hours}       &  3 (Theoretical) & \textbf{ECTS Credit Hours}       &  6.9 \\ \hline
\end{tabularx}

\begin{tabularx}{\textwidth}{|l|X|}
\hline
\textbf{Pre-requisite}      &  -- \\ \hline
\textbf{Co-requisite}       &  -- \\ \hline
\textbf{Text Book(s)}      & \begin{minipage}{.70\textwidth}
					\begin{itemize} \itemsep-0.4em
						\vspace{3mm}
						\item James Stewart, Single Variable Calculus: Concepts and Contexts, 4th edition, Cengage Learning, 2009.
						\item George Simmons, Calculus with Analytic Geometry. 2nd Edition, McGraw-Hill Science/Engineering/Math, 1996.
						\item Tom Apostol, Calculus, Vol. 1: One-Variable Calculus, with an Introduction to Linear Algebra, Willey, 2nd Edition, 1991.
						\vspace{3mm}
					\end{itemize}
				\end{minipage}  \\ \hline
\textbf{Course Description} & \begin{minipage}{.70\textwidth}
					\vspace{3mm}
					
					This is a two course sequence in the differential and integral calculus of functions of one independent variable. Topics include the basic analytic geometry of graphs of functions, and their limits, integrals and derivatives, including the Fundamental Theorem of Calculus. Also, some applications of the integral, like arc length and volumes of solids with rotational symmetry, are discussed. Applications to the physical sciences and engineering will be a focus of this course, as this sequence of courses is designed to meet the needs of students in these disciplines.
					\newline
					
					Tangent lines; limits and continuity; differentiation: definition, basic rules, chain rule, rules for trig, exp and log functions; implicit differentiation; rates of change, max-min, related rates problems; 2nd derivative test; curve sketching; linear approximation and differentials; L'Hospital's rule; integration: definition, anti differentiation, area; simple substitution; volumes of solids by cross sections and shells; work; average value of a function.
					
					\vspace{3mm}
					\end{minipage} \\ \hline
\end{tabularx}
\end{minipage}

\bigskip
\bigskip

 \begin{minipage}{\textwidth}
\begin{tabularx}{\textwidth}{|l|X|l|X|}
\hline
\textbf{Course Title}       &   Calculus 2 & \textbf{Course Code}       &   5712094 \\ \hline
\textbf{Credit Hours}       &  3 (Theoretical) & \textbf{ECTS Credit Hours}       &  6.9 \\ \hline
\end{tabularx}

\begin{tabularx}{\textwidth}{|l|X|}
\hline
\textbf{Pre-requisite}      &  Calculus 1 \\ \hline
\textbf{Co-requisite}       &  -- \\ \hline
\textbf{Text Book(s)}      & \begin{minipage}{.70\textwidth}
					\begin{itemize} \itemsep-0.4em
						\vspace{3mm}
						\item James Stewart, Multivariable Calculus. Cengage Learning, 7th Edition, 2011.
						\item Tom Apostol, Calculus, Vol. 2: Multi-Variable Calculus and Linear Algebra with Applications to Differential Equations and Probability. Wiley, 1969.
						\item George Simmons, Calculus with Analytic Geometry, 2nd Edition, McGraw-Hill Science/Engineering/Math, 1996.
						\vspace{3mm}
					\end{itemize}
				\end{minipage}  \\ \hline
\textbf{Course Description} & \begin{minipage}{.70\textwidth}
					\vspace{3mm}
					
					This is the second of a two course sequence in the differential and integral calculus of functions of one independent variable. Topics include the basic and advanced techniques of integration, analytic geometry of graphs of functions, and their limits, integrals and derivatives, including the Fundamental Theorem of Calculus. Also, some applications of the integral, like arc length and volumes of solids with rotational symmetry, are discussed. Applications to the physical sciences and engineering will be a focus of this course, as this sequence of courses is designed to meet the needs of students in these disciplines.
					\newline
					
					Techniques of integration, including integration by parts, simple trig substitutions, partial fractions. Basic numerical integration; improper integrals; arc length; area of surface of revolution. Separable differential equations, Euler's method, exponential growth and decay. Parametric curves and polar coordinates. Review of conic sections. Sequences and series, comparison and ratio tests, Taylor series and polynomials. Vectors in three dimensions, dot product, cross product; lines, vector valued functions of a scalar.

					\vspace{3mm}
					\end{minipage} \\ \hline
\end{tabularx}
\end{minipage}


\bigskip
\bigskip

\begin{minipage}{\textwidth}
\begin{tabularx}{\textwidth}{|l|X|l|X|}
\hline
\textbf{Course Title}       &  Physics 1 (Heat and Mechanics) & \textbf{Course Code}       &  4210113 \\ \hline
\textbf{Credit Hours}       &  3 (Theoretical) & \textbf{ECTS Credit Hours}       &   6.9 \\ \hline
\end{tabularx}

\begin{tabularx}{\textwidth}{|l|X|}
\hline
\textbf{Pre-requisite}      & Calculus 1 \\ \hline
\textbf{Co-requisite}       &  -- \\ \hline
\textbf{Text Book(s)}      & \begin{minipage}{.70\textwidth}
					\begin{itemize} \itemsep-0.4em
						\vspace{3mm}
						\item D. Halliday, R. Resnick, and J. Walker, Fundamentals of Physics. 9th Edition, Wiley, 2010.
						\vspace{3mm}
					\end{itemize}
				\end{minipage}  \\ \hline
\textbf{Course Description} & \begin{minipage}{.70\textwidth}
					\vspace{3mm}
					This is the first semester of a two-semester sequence of calculus-based introductory physics. This course uses calculus. Topics include kinematics, dynamics, rotational motion, gravitation, conservation laws of momentum and energy, thermal physics, and periodic motion.  Optional topics include fluids and thermodynamics.  This course meets requirements for students majoring in engineering, mathematics, computer science, or the sciences.

					\vspace{3mm}
					\end{minipage} \\ \hline
\end{tabularx}
\end{minipage}


\bigskip
\bigskip


\begin{minipage}{\textwidth}
\begin{tabularx}{\textwidth}{|l|X|l|X|}
\hline
\textbf{Course Title}       &   Physics 2 (Electricity and Magnetism) & \textbf{Course Code}       &  4210115 \\ \hline
\textbf{Credit Hours}       &   3 (Theoretical) & \textbf{ECTS Credit Hours}       &   6.9 \\ \hline
\end{tabularx}

\begin{tabularx}{\textwidth}{|l|X|}
\hline
\textbf{Pre-requisite}      &  Calculus 1 \\ \hline
\textbf{Co-requisite}       &  -- \\ \hline
\textbf{Text Book(s)}      & \begin{minipage}{.70\textwidth}
					\begin{itemize} \itemsep-0.4em
						\vspace{3mm}
						\item D. Halliday, R. Resnick, and J. Walker, Fundamentals of Physics. 9th Edition, Wiley, 2010.
						\vspace{3mm}
					\end{itemize}
				\end{minipage}  \\ \hline
\textbf{Course Description} & \begin{minipage}{.70\textwidth}
					\vspace{3mm}
					
This course is a continuation of Physics 1.  This course uses calculus. Topics include wave phenomena, electricity, magnetism, an introduction to Maxwell's equations, electromagnetic waves, and optics. This course meets requirements for students majoring in engineering, mathematics, computer science, or the sciences
					\vspace{3mm}
					\end{minipage} \\ \hline
\end{tabularx}
\end{minipage}

\bigskip
\bigskip


\begin{minipage}{\textwidth}
\begin{tabularx}{\textwidth}{|l|X|l|X|}
\hline
\textbf{Course Title}       &    Engineering Probability and Statistics & \textbf{Course Code}       & 1110261  \\ \hline
\textbf{Credit Hours}       &  3 (Theoretical) & \textbf{ECTS Credit Hours}       &  6.9 \\ \hline
\end{tabularx}

\begin{tabularx}{\textwidth}{|l|X|}
\hline
\textbf{Pre-requisite}      &  Calculus 2 \\ \hline
\textbf{Co-requisite}       &  -- \\ \hline
\textbf{Text Book(s)}      & \begin{minipage}{.70\textwidth}
					\begin{itemize} \itemsep-0.4em
						\vspace{3mm}
						\item Alberto Leon-Garcia, Probability, Statistics, and Random Processes for Electrical Engineering,. Prentice Hall, 3rd Edition, 2008.
						\item Ronald E. Walpole, Raymond H. Myers, Sharon L. Myers, and Keying E. Ye, Probability and Statistics for Engineers and Scientists. Pearson, 9th Edition, 2011.
						\vspace{3mm}
					\end{itemize}
				\end{minipage}  \\ \hline
\textbf{Course Description} & \begin{minipage}{.70\textwidth}
					\vspace{3mm}
					An introduction to probability theory and statistics, with an emphasis on solving problems in computer science and engineering. Probability and statistics is an important foundation for computer science fields such as machine learning, artificial intelligence, computer graphics, randomized algorithms, image processing, and scientific simulations. Topics in probability include discrete and continuous random variables, probability distributions, sums and functions of random variables, the law of large numbers, and the central limit theorem, moments, moment generating function, Markov and Chebyshev inequalities. Topics in statistics include sample mean and variance, estimating distributions, correlation, regression, and hypothesis testing. Beyond the fundamentals, this course will also focus on modern computational methods such as simulation and the bootstrap.

					\vspace{3mm}
					\end{minipage} \\ \hline
\end{tabularx}
\end{minipage}


\bigskip
\bigskip



\begin{minipage}{\textwidth}
\begin{tabularx}{\textwidth}{|l|X|l|X|}
\hline
\textbf{Course Title}       &    Differential Equations & \textbf{Course Code}       &  1110203 \\ \hline
\textbf{Credit Hours}       &   3 (Theoretical) & \textbf{ECTS Credit Hours}       &   6.9 \\ \hline
\end{tabularx}

\begin{tabularx}{\textwidth}{|l|X|}
\hline
\textbf{Pre-requisite}      &  Calculus 1 \\ \hline
\textbf{Co-requisite}       &  -- \\ \hline
\textbf{Text Book(s)}      & \begin{minipage}{.70\textwidth}
					\begin{itemize} \itemsep-0.4em
						\vspace{3mm}
						\item Yunus Cengel and William Palm, Differential Equations for Engineers and Scientists. McGraw-Hill Science/Engineering/Math, 1th Edition, 2012.
						\vspace{3mm}
					\end{itemize}
				\end{minipage}  \\ \hline
\textbf{Course Description} & \begin{minipage}{.70\textwidth}
					\vspace{3mm}
					This course includes the study of first order differential equations, higher order linear
					differential equations, Laplace transforms, numerical methods, boundary value and initial value
					problems, qualitative analysis of solutions, and applications of differential equations.

					\vspace{3mm}
					\end{minipage} \\ \hline
\end{tabularx}
\end{minipage}


\bigskip
\bigskip


\begin{minipage}{\textwidth}
\begin{tabularx}{\textwidth}{|l|X|l|X|}
\hline
\textbf{Course Title}       & Computer Workshop  & \textbf{Course Code}       &  1912027 \\ \hline
\textbf{Credit Hours}       &   1 (Practical) & \textbf{ECTS Credit Hours}       &  6.9 \\ \hline
\end{tabularx}

\begin{tabularx}{\textwidth}{|l|X|}
\hline
\textbf{Pre-requisite}      &  Fundamentals of Computer Programming \\ \hline
\textbf{Co-requisite}       &  -- \\ \hline
\textbf{Text Book(s)}      & -- \\ \hline
\textbf{Course Description} & \begin{minipage}{.70\textwidth}
					\vspace{3mm}
					Familiarity with accessory systems such as card reader, printers, magnetic tape, disc and console,
					manner of work with terminal, familiarity with compilers and editors,
					familiarity with computer organization of a center, familiarity with prepared software packages such as		
					database, spreadsheet, lotus, familiarity with the important programs of system such as sort, merge,
					creation and copy of files etc., familiarity with usage manner of an operating system of microcomputer

					\vspace{3mm}
					\end{minipage} \\ \hline
\end{tabularx}
\end{minipage}


\bigskip
\bigskip



\begin{minipage}{\textwidth}
\begin{tabularx}{\textwidth}{|l|X|l|X|}
\hline
\textbf{Course Title}       & Electric Circuits  & \textbf{Course Code}       &  1110136 \\ \hline
\textbf{Credit Hours}       &  3 (Theoretical) & \textbf{ECTS Credit Hours}       &  6.9 \\ \hline
\end{tabularx}

\begin{tabularx}{\textwidth}{|l|X|}
\hline
\textbf{Pre-requisite}      &  Differential Equations \\ \hline
\textbf{Co-requisite}       &  -- \\ \hline
\textbf{Text Book(s)}      & \begin{minipage}{.70\textwidth}
					\begin{itemize} \itemsep-0.4em
						\vspace{3mm}
						\item W. H. Hayt, J. E. Kemmerly, and S. M. Durbin, Engineering Circuit Analysis, McGraw Hill. 8th Edition, 2012.
						\vspace{3mm}
					\end{itemize}
				\end{minipage}  \\ \hline
\textbf{Course Description} & \begin{minipage}{.70\textwidth}
					\vspace{3mm}
					Fundamental concepts in electrical circuits; circuit analysis and network theorems; linearity and superposition; series/parallel combinations of R, L, and C circuits; sinusoidal forcing; complex frequency and Bode plots; mutual inductance and transformers; two port networks.

					\vspace{3mm}
					\end{minipage} \\ \hline
\end{tabularx}
\end{minipage}


\bigskip
\bigskip



\begin{minipage}{\textwidth}
\begin{tabularx}{\textwidth}{|l|X|l|X|}
\hline
\textbf{Course Title}       &   Physics Laboratory 2 (Electiricy and Magnetism) & \textbf{Course Code}       &  4210116 \\ \hline
\textbf{Credit Hours}       &  1  (Laboratory) & \textbf{ECTS Credit Hours}       &  4.6  \\ \hline
\end{tabularx}

\begin{tabularx}{\textwidth}{|l|X|}
\hline
\textbf{Pre-requisite}      &  Physics 2 (Electricity and Magnetism) \\ \hline
\textbf{Co-requisite}       &  -- \\ \hline
\textbf{Text Book(s)}      & \begin{minipage}{.70\textwidth}
					\begin{itemize} \itemsep-0.4em
						\vspace{3mm}
						\item D. Haliday, R. Resnick, and J. Walker, Fundamentals of Physics. 9th Edition, Wiley, 2010.
						\vspace{3mm}
					\end{itemize}
				\end{minipage}  \\ \hline
\textbf{Course Description} & \begin{minipage}{.70\textwidth}
					\vspace{3mm}
					
					--
					\vspace{3mm}
					\end{minipage} \\ \hline
\end{tabularx}
\end{minipage}


\bigskip
\bigskip



\begin{minipage}{\textwidth}
\begin{tabularx}{\textwidth}{|l|X|l|X|}
\hline
\textbf{Course Title}       &   Fundamentals of Computer Programming & \textbf{Course Code}       &  1912011 \\ \hline
\textbf{Credit Hours}       &  3  (Theoretical) & \textbf{ECTS Credit Hours}       &  6.9 \\ \hline
\end{tabularx}

\begin{tabularx}{\textwidth}{|l|X|}
\hline
\textbf{Pre-requisite}      &  -- \\ \hline
\textbf{Co-requisite}       &  -- \\ \hline
\textbf{Text Book(s)}      & \begin{minipage}{.70\textwidth}
					\begin{itemize} \itemsep-0.4em
						\vspace{3mm}
						\item Harvey Deitel and Paul Deitel, C++ How to Program, 8 Edition, Pearson,2015.
						\vspace{3mm}
					\end{itemize}
				\end{minipage}  \\ \hline
\textbf{Course Description} & \begin{minipage}{.70\textwidth}
					\vspace{3mm}
					An introduction to computer programming using a high level programming language. Concepts and topics
					covered include the basic components of algorithms (primitive operations, variables, sequencing operations,
					conditionals/branching, repetition/loops, and subroutines/functions), problem decomposition, abstraction, testing
					and debugging, pseudo-code, file based input and output, use of a modern development environment, good
					coding style, pointers/references, and basic data structures (arrays, records/structs, objects).					

					\vspace{3mm}
					\end{minipage} \\ \hline
\end{tabularx}
\end{minipage}


\bigskip
\bigskip




\begin{minipage}{\textwidth}
\begin{tabularx}{\textwidth}{|l|X|l|X|}
\hline
\textbf{Course Title}       &  Discrete Mathematics & \textbf{Course Code}       &  1912027 \\ \hline
\textbf{Credit Hours}       &  3 (Theoretical) & \textbf{ECTS Credit Hours}       & 6.9  \\ \hline
\end{tabularx}

\begin{tabularx}{\textwidth}{|l|X|}
\hline
\textbf{Pre-requisite}      &  -- \\ \hline
\textbf{Co-requisite}       &  Calculus I and Fundamentals of Computer Programming \\ \hline
\textbf{Text Book(s)}      & \begin{minipage}{.70\textwidth}
					\begin{itemize} \itemsep-0.4em
						\vspace{3mm}
						\item R. P. Grimaldi, Discrete and Combinatorial Mathematics: An Applied Introduction. 5th Edition, Addison-Wesley Inc., 2004
						\item K. H. Rosen, Discrete Mathematics and Its Applications. 6th Edition, McGraw Hill Inc., 2007.
						\vspace{3mm}
					\end{itemize}
				\end{minipage}  \\ \hline
\textbf{Course Description} & \begin{minipage}{.70\textwidth}
					\vspace{3mm}
					\textbf{Introduction}: mathematical logic, algebra of expressions, well-structured formula, a review of theory of
					sets, proving methods. \newline
					\textbf{Relations and functions}: dual relations, compatibility and equivalence relations, relations
					representation matrix, relations graph, functions, surjective functions, one to one functions, recursive
					relations, solving recursive functions, generating function.\newline
					\textbf{Algebraic structures}: semi-groups and monoids, grammars and languages, Polish marking, groups,
					homomorphism, isomorphism, lattices, boolean algebra, Carnot’s table, grammar, grammar as an
					example of monoids\newline
					\textbf{Combinational analysis}: pigeon hole principle, an introduction to combinational algorithms, recursive
					functions and their application.\newline
					\textbf{Graph theory}: directed graphs, undirected graphs, Eulerian path and Hamiltonian path, optimal paths,
					algorithm finding of optimal paths, connected graphs, matrix of relation and related theorems, graph
					applications in activities analysis.\newline
					\textbf{Trees}: minimal surjective trees, mensuration of tree, application of trees, algebraic expressions and
					representation of their trees.					

					\vspace{3mm}
					\end{minipage} \\ \hline
\end{tabularx}
\end{minipage}


\bigskip
\bigskip



\begin{minipage}{\textwidth}
\begin{tabularx}{\textwidth}{|l|X|l|X|}
\hline
\textbf{Course Title}       &  Advanced Programming & \textbf{Course Code}       &  1912002 \\ \hline
\textbf{Credit Hours}       &   3 (Theoretical) & \textbf{ECTS Credit Hours}       &  6.9 \\ \hline
\end{tabularx}

\begin{tabularx}{\textwidth}{|l|X|}
\hline
\textbf{Pre-requisite}      &  Fundamentals of Computer Programming \\ \hline
\textbf{Co-requisite}       &  -- \\ \hline
\textbf{Text Book(s)}      & \begin{minipage}{.70\textwidth}
					\begin{itemize} \itemsep-0.4em
						\vspace{3mm}
						\item P. Deitel and H. Deitel, Java: How to Program,. 9th Edition, Prentice Hall Inc., 2011.
						\item B. Eckel, Thinking in Java. 4th Edition, Prentice Hall Inc., 2006.
						\vspace{3mm}
					\end{itemize}
				\end{minipage}  \\ \hline
\textbf{Course Description} & \begin{minipage}{.70\textwidth}
					\vspace{3mm}
					\begin{enumerate}
						  \item Top-Down design approach.
						  \item Basic Object-Oriented principles: modeling based on real world, abstraction.
						  \item Object-Oriented programming components: Object, Class, Method, Constructor.
						  \item Inheritance and Polymorphism.
						  \item Memory management -- Introduction to dynamic memory allocation.
						  \item Generic programming.
						  \item Exception Handling.
						  \item I/O handling.
						  \item Collections.
						  \item Graphical User Interface programming.
						  \item Introduction to concurrent and parallel programming.
						  \item Debug and test tools.
					\end{enumerate}					

					\vspace{3mm}
					\end{minipage} \\ \hline
\end{tabularx}
\end{minipage}


\bigskip
\bigskip



\begin{minipage}{\textwidth}
\begin{tabularx}{\textwidth}{|l|X|l|X|}
\hline
\textbf{Course Title}       &   Data Structures & \textbf{Course Code}       & 1912003  \\ \hline
\textbf{Credit Hours}       &  3 (Theoretical) & \textbf{ECTS Credit Hours}       &  6.9  \\ \hline
\end{tabularx}

\begin{tabularx}{\textwidth}{|l|X|}
\hline
\textbf{Pre-requisite}      &  Advanced Programming, Discrete Mathematics \\ \hline
\textbf{Co-requisite}       &  -- \\ \hline
\textbf{Text Book(s)}      & \begin{minipage}{.70\textwidth}
					\begin{itemize} \itemsep-0.4em
						\vspace{3mm}
						\item T. Cormen, C. Leiserson, and R. Rivest,. Introduction to Algorithms. McGraw-Hill Inc., 2001.
					           \item E. Horowitz and S. Sahni, Fundamentals of Computer Algorithms, Computer Science Press, Rockville, MD, 1984.
						\vspace{3mm}
					\end{itemize}
				\end{minipage}  \\ \hline
\textbf{Course Description} & \begin{minipage}{.70\textwidth}
					\vspace{3mm}
					Covers the design, analysis, and implementation of data structures and algorithms to solve engineering
					problems using an object‐oriented programming language. Topics include elementary data structures,
					(including arrays, stacks, queues, and lists), advanced data structures (including trees and graphs),the
					algorithms used to manipulate these structures, and their application to solving practical engineering
					problems.

					\vspace{3mm}
					\end{minipage} \\ \hline
\end{tabularx}
\end{minipage}


\bigskip
\bigskip



\begin{minipage}{\textwidth}
\begin{tabularx}{\textwidth}{|l|X|l|X|}
\hline
\textbf{Course Title}       &  Logic Circuits & \textbf{Course Code}       &  1110239 \\ \hline
\textbf{Credit Hours}       &   (Theoretical) & \textbf{ECTS Credit Hours}       &  6.9  \\ \hline
\end{tabularx}

\begin{tabularx}{\textwidth}{|l|X|}
\hline
\textbf{Pre-requisite}      &  -- \\ \hline
\textbf{Co-requisite}       &  Discrete Mathematics \\ \hline
\textbf{Text Book(s)}      & \begin{minipage}{.70\textwidth}
					\begin{itemize} \itemsep-0.4em
						\vspace{3mm}
						\item S. Brown and Z. Vranesic, Fundamentals of Digital Logic with Verilog Design. 3rd Edition, McGraw-Hill, 2009.
						\item C. H. Roth and L. L. Kinney, Fundamentals of Logic Design. 5th Edition, 2005.
						\vspace{3mm}
					\end{itemize}
				\end{minipage}  \\ \hline
\textbf{Course Description} & \begin{minipage}{.70\textwidth}
					\vspace{3mm}
					This course provides the student with a foundation in the fundamentals of digital logic design and
					computer logic circuits. Both combinational and sequential logic circuits are covered in this course. The
					emphasis is on the use of Boolean algebra and basic logic gates to build cost effective complex logic
					circuits. Topics include: Number systems, Binary arithmetic, Codes, Logic gates, Boolean algebra and
					simplifications, Half adders, Full adders, Decoders, Encoders, Multiplexers, Latches, Flip-Flops,
					Counters, Shift Registers, Memory circuits, and ALU.

					\vspace{3mm}
					\end{minipage} \\ \hline
\end{tabularx}
\end{minipage}


\bigskip
\bigskip



\begin{minipage}{\textwidth}
\begin{tabularx}{\textwidth}{|l|X|l|X|}
\hline
\textbf{Course Title}       &  The Theory of Formal Languages and Automata & \textbf{Course Code}       &  1912005 \\ \hline
\textbf{Credit Hours}       & 3  (Theoretical) & \textbf{ECTS Credit Hours}       &  6.9 \\ \hline
\end{tabularx}

\begin{tabularx}{\textwidth}{|l|X|}
\hline
\textbf{Pre-requisite}      &  Data Structures\\ \hline
\textbf{Co-requisite}       &  -- \\ \hline
\textbf{Text Book(s)}      & \begin{minipage}{.70\textwidth}
					\begin{itemize} \itemsep-0.4em
						\vspace{3mm}
						\item P. Linz, An Introduction to Formal Languages and Automata. 5th Edition, Jones and Barlett Publishers, 2011.
						\item M. Sipser, Introduction to the theory of computation. 2nd Edition, PWS Publishing Company, 2006.
						\vspace{3mm}
					\end{itemize}
				\end{minipage}  \\ \hline
\textbf{Course Description} & \begin{minipage}{.70\textwidth}
					\vspace{3mm}
					The course introduces some fundamental concepts in automata theory and formal
					languages including grammar, finite automaton, regular expression, formal language, push-down automaton,
					and Turing machine. Not only do they form basic models of computation, they are also the foundation of
					many branches of computer science, e.g. compilers, software engineering, concurrent systems, etc. The
					properties of these models will be studied and various rigorous techniques for analyzing and comparing
					them will be discussed, by using both formalism and examples.

					\vspace{3mm}
					\end{minipage} \\ \hline
\end{tabularx}
\end{minipage}


\bigskip
\bigskip


\begin{minipage}{\textwidth}
\begin{tabularx}{\textwidth}{|l|X|l|X|}
\hline
\textbf{Course Title}       &   Technical English & \textbf{Course Code}       &  1910039 \\ \hline
\textbf{Credit Hours}       &   2 (Theoretical) & \textbf{ECTS Credit Hours}       &  4.6 \\ \hline
\end{tabularx}

\begin{tabularx}{\textwidth}{|l|X|}
\hline
\textbf{Pre-requisite}      &  General English \\ \hline
\textbf{Co-requisite}       &  -- \\ \hline
\textbf{Text Book(s)}      & \begin{minipage}{.70\textwidth}
					\begin{itemize} \itemsep-0.4em
						\vspace{3mm}
						\item TED group scientific lectures
						\item IEEE Spectrum Magazine
						\vspace{3mm}
					\end{itemize}
				\end{minipage}  \\ \hline
\textbf{Course Description} & \begin{minipage}{.70\textwidth}
					\vspace{3mm}
										
					Introduction to basic concepts and grammar relevant to computer science, used vocabularies in
					software, hardware, internet, information networks.
					Familiarity with common messages in operating systems and software installation and programming
					languages and abbreviation in email and chat and search engines.
					Texts translation relevant to computer.
					\vspace{3mm}
					\end{minipage} \\ \hline
\end{tabularx}
\end{minipage}


\bigskip
\bigskip



\begin{minipage}{\textwidth}
\begin{tabularx}{\textwidth}{|l|X|l|X|}
\hline
\textbf{Course Title}       &  Research and Technical Presentation & \textbf{Course Code}       &  1912029 \\ \hline
\textbf{Credit Hours}       &  2 (Theoretical) & \textbf{ECTS Credit Hours}       &   4.6 \\ \hline
\end{tabularx}

\begin{tabularx}{\textwidth}{|l|X|}
\hline
\textbf{Pre-requisite}      &  Technical English \\ \hline
\textbf{Co-requisite}       &  -- \\ \hline
\textbf{Text Book(s)}      & -- \\ \hline
\textbf{Course Description} & \begin{minipage}{.70\textwidth}
					\vspace{3mm}
					Different types of scientific and technical subjects (letters, reports, pamphlets, manual and etc.), 
					common points in all scientific and technical writings: specifying the objective of writing and its
					eventual readers, organizing the subjects, abstract of essay together with report, the role of a good
					introduction, dividing the subjects into parts and chapters, discussion and conclusion, preparing source
					and reference index, attachments, preparing the pictures and diagrams and tables. Important points in
					translation of scientific and technical subjects, writing style, marking and its importance, preparing final
					format of writing by typing machine or computer, foot-article, notes and other lateral subjects, an
					introduction to research methods, presenting subjects orally, effective use of audio-visual devices, the
					rules and process of drawing up graduation diploma including the main parts of thesis and details of
					each part, preparing and presenting a scientific essay (as assignment)

					\vspace{3mm}
					\end{minipage} \\ \hline
\end{tabularx}
\end{minipage}


\bigskip
\bigskip



\begin{minipage}{\textwidth}
\begin{tabularx}{\textwidth}{|l|X|l|X|}
\hline
\textbf{Course Title}       &   Engineering Mathematics & \textbf{Course Code}       & 1110001  \\ \hline
\textbf{Credit Hours}       &  3 (Theoretical) & \textbf{ECTS Credit Hours}       &  6.9  \\ \hline
\end{tabularx}

\begin{tabularx}{\textwidth}{|l|X|}
\hline
\textbf{Pre-requisite}      &   Calculus 2, Differential Equations \\ \hline
\textbf{Co-requisite}       &  -- \\ \hline
\textbf{Text Book(s)}      & \begin{minipage}{.70\textwidth}
					\begin{itemize} \itemsep-0.4em
						\vspace{3mm}
						\item E. Kreyszig, Advanced Engineering Mathematics, 10th Edition, Wiley, 2011.
						\item C. R. Wylie, Advanced Engineering Mathematics, 6th Edition, McGraw-Hill, 1995.
						\vspace{3mm}
					\end{itemize}
				\end{minipage}  \\ \hline
\textbf{Course Description} & \begin{minipage}{.70\textwidth}
					\vspace{3mm}
					This course will provide an overview of the salient math topics most heavily used
					in the core sophomore-level engineering courses. These include algebraic manipulation of
					engineering equations, trigonometry, vectors and complex numbers, sinusoidal and harmonic
					signals, systems of equations and matrices, differentiation, integration and differential equations.
					All math topics will be presented within the context of an engineering application, and reinforced
					through extensive examples of their use in the core engineering courses.

					\vspace{3mm}
					\end{minipage} \\ \hline
\end{tabularx}
\end{minipage}


\bigskip
\bigskip




\begin{minipage}{\textwidth}
\begin{tabularx}{\textwidth}{|l|X|l|X|}
\hline
\textbf{Course Title}       &  Computer Architecture & \textbf{Course Code}       &   1914002\\ \hline
\textbf{Credit Hours}       &  3 (Theoretical) & \textbf{ECTS Credit Hours}       & 6.9  \\ \hline
\end{tabularx}

\begin{tabularx}{\textwidth}{|l|X|}
\hline
\textbf{Pre-requisite}      &  Logic Circuits \\ \hline
\textbf{Co-requisite}       &  -- \\ \hline
\textbf{Text Book(s)}      & \begin{minipage}{.70\textwidth}
					\begin{itemize} \itemsep-0.4em
						\vspace{3mm}
						\item D. A. Patterson and J. L. Hennessy, Computer Organization and Design: The Hardware/Software Interface, 4th Edition, Morgan Kaufmann Publishers Inc., 2010.
						\vspace{3mm}
					\end{itemize}
				\end{minipage}  \\ \hline
\textbf{Course Description} & \begin{minipage}{.70\textwidth}
					\vspace{3mm}
					Fundamentals of computer design; quantifying cost and performance; instruction set architecture; program behavior and measurement of instruction set use;
					 processor datapaths and control; pipelining, handling pipeline hazards; memory hierarchies and performance;
					 I/O devices, controllers and drivers; I/O and system performance.

					\vspace{3mm}
					\end{minipage} \\ \hline
\end{tabularx}
\end{minipage}


\bigskip
\bigskip



\begin{minipage}{\textwidth}
\begin{tabularx}{\textwidth}{|l|X|l|X|}
\hline
\textbf{Course Title}       &   Operating Systems & \textbf{Course Code}       &  1914009 \\ \hline
\textbf{Credit Hours}       &  3  (Theoretical) & \textbf{ECTS Credit Hours}       &   6.9 \\ \hline
\end{tabularx}

\begin{tabularx}{\textwidth}{|l|X|}
\hline
\textbf{Pre-requisite}      &   Data Structures, Computer Architecture \\ \hline
\textbf{Co-requisite}       &  -- \\ \hline
\textbf{Text Book(s)}      & \begin{minipage}{.70\textwidth}
					\begin{itemize} \itemsep-0.4em
						\vspace{3mm}
						\item P. Silberschatz, B. Galvin, and G. Gagne, Operating System Concepts. 8th Edition, John Wiley Inc., 2010.
						\vspace{3mm}
					\end{itemize}
				\end{minipage}  \\ \hline
\textbf{Course Description} & \begin{minipage}{.70\textwidth}
					\vspace{3mm}
					A fundamental overview of operating systems. Topics covered include: Operating system structures, processes, 
					process synchronization, deadlocks, CPU scheduling, memory management, file systems, secondary storage management.

					\vspace{3mm}
					\end{minipage} \\ \hline
\end{tabularx}
\end{minipage}


\bigskip
\bigskip



\begin{minipage}{\textwidth}
\begin{tabularx}{\textwidth}{|l|X|l|X|}
\hline
\textbf{Course Title}       &   Design of Algorithms & \textbf{Course Code}       &  1912004 \\ \hline
\textbf{Credit Hours}       &  3  (Theoretical) & \textbf{ECTS Credit Hours}       &  6.9 \\ \hline
\end{tabularx}

\begin{tabularx}{\textwidth}{|l|X|}
\hline
\textbf{Pre-requisite}      &   Data Structures \\ \hline
\textbf{Co-requisite}       &  -- \\ \hline
\textbf{Text Book(s)}      & \begin{minipage}{.70\textwidth}
					\begin{itemize} \itemsep-0.4em
						\vspace{3mm}
						\item T. Cormen, C. Leisorson, and R. Rivest. Introduction to Algorithms, 3rd Edition, McGraw-Hill Inc., 2001.
						\vspace{3mm}
					\end{itemize}
				\end{minipage}  \\ \hline
\textbf{Course Description} & \begin{minipage}{.70\textwidth}
					\vspace{3mm}
					This course is concerned with issues that arise in the design of algorithms for solving computational problems.
					In the first part methods a number of standard algorithm design paradigms are presented and example applications of these examined.
					In the second part of the course some theoretical issues in algorithm design are examined: the concepts of computability and computational tractability
					are introduced and some examples of computational problems with no feasible algorithmic solution are presented.

					\vspace{3mm}
					\end{minipage} \\ \hline
\end{tabularx}
\end{minipage}


\bigskip
\bigskip


\begin{minipage}{\textwidth}
\begin{tabularx}{\textwidth}{|l|X|l|X|}
\hline
\textbf{Course Title}       &  Computer Aided Digital System Design & \textbf{Course Code}       &  1914004 \\ \hline
\textbf{Credit Hours}       &  3 (Theoretical) & \textbf{ECTS Credit Hours}       & 6.9  \\ \hline
\end{tabularx}

\begin{tabularx}{\textwidth}{|l|X|}
\hline
\textbf{Pre-requisite}      &  Computer Architecture \\ \hline
\textbf{Co-requisite}       &  -- \\ \hline
\textbf{Text Book(s)}      & \begin{minipage}{.70\textwidth}
					\begin{itemize} \itemsep-0.4em
						\vspace{3mm}
						\item S. Palnitkar, Verilog HDL: A Guide to Digital Design and Synthesis. SunSoft Press, 2nd Edition, 2003.
						\item C. Maxfield, The Design Warrior's Guide to FPGAs: Devices, Tools and Flows. Elsevier Publication, 2004.
						\vspace{3mm}
					\end{itemize}
				\end{minipage}  \\ \hline
\textbf{Course Description} & \begin{minipage}{.70\textwidth}
					\vspace{3mm}
					
					This course covers the systematic design of advanced digital systems using field-programmable gate arrays (FPGAs). The emphasis is on top-down design starting with a software application, and translating it to high-level models using a hardware description language (such as VHDL or Verilog). The course will focus on design for high-performance computing applications using streaming architectures. We will first review in detail the basic building blocks of FPGA programming. Second, we focus on architecture, design methodologies, best design practices, and optimization techniques for performance (frequency, latency, area, power, etc).

					\vspace{3mm}
					\end{minipage} \\ \hline
\end{tabularx}
\end{minipage}


\bigskip
\bigskip


\begin{minipage}{\textwidth}
\begin{tabularx}{\textwidth}{|l|X|l|X|}
\hline
\textbf{Course Title}       &  Signals and Systems & \textbf{Course Code}       &   1110256 \\ \hline
\textbf{Credit Hours}       & 3  (Theoretical) & \textbf{ECTS Credit Hours}       &  6.9 \\ \hline
\end{tabularx}

\begin{tabularx}{\textwidth}{|l|X|}
\hline
\textbf{Pre-requisite}      &  Engineering Mathematics \\ \hline
\textbf{Co-requisite}       &  -- \\ \hline
\textbf{Text Book(s)}      & \begin{minipage}{.70\textwidth}
					\begin{itemize} \itemsep-0.4em
						\vspace{3mm}
						\item A. V. Oppenheim, A. S. Willsky, and S. H. Nawab, Signals and Systems. 2nd Edition, Prentice-Hall, 1996.
						\item R. E. Ziemer, W. H. Tranter, and S. R. Fannin, Signals and Systems, Continuous and Discrete. 4th Edition, Prentice-Hall, 1998.
						\vspace{3mm}
					\end{itemize}
				\end{minipage}  \\ \hline
\textbf{Course Description} & \begin{minipage}{.70\textwidth}
					\vspace{3mm}
					Continuous signals and systems: block diagrams, linearity, causality, stability and time-invariance, linear time-invariant (LTI) systems, impulse response; Convolution sum and integral; Convolution and correlation; introduction to Stochastic Signals. Fourier techniques in signals and systems: Fourier series and transform of signals; Frequency response of continuous time LTI circuits and systems; Fourier transforms and continuous spectra; Applications, correlation and power spectrum. 

					\vspace{3mm}
					\end{minipage} \\ \hline
\end{tabularx}
\end{minipage}


\bigskip
\bigskip



\begin{minipage}{\textwidth}
\begin{tabularx}{\textwidth}{|l|X|l|X|}
\hline
\textbf{Course Title}       &   Microprocessors and Assembly Language & \textbf{Course Code}       &  1914043 \\ \hline
\textbf{Credit Hours}       &   3 (Theoretical) & \textbf{ECTS Credit Hours}       &  6.9 \\ \hline
\end{tabularx}

\begin{tabularx}{\textwidth}{|l|X|}
\hline
\textbf{Pre-requisite}      &  Computer Architecture \\ \hline
\textbf{Co-requisite}       &  -- \\ \hline
\textbf{Text Book(s)}      & \begin{minipage}{.70\textwidth}
					\begin{itemize} \itemsep-0.4em
						\vspace{3mm}
						\item John Uffenbeck, The 8086/8088 Family: Design, Programming, and Interfacing. 3rd Edition, Prentice Hall, 2001.
						\vspace{3mm}
					\end{itemize}
				\end{minipage}  \\ \hline
\textbf{Course Description} & \begin{minipage}{.70\textwidth}
					\vspace{3mm}
					Concepts of assembly language and the machine representation of instructions and
					data of a modern digital computer are presented. Students will have the opportunity to study machine
					addressing, stack operations, subroutines, and programmed and interrupt driven I/O. Also, basic concepts of
					machine organization are studied. This will include computer architecture at the register level and micro-operation components of instructions.

					\vspace{3mm}
					\end{minipage} \\ \hline
\end{tabularx}
\end{minipage}


\bigskip
\bigskip



\begin{minipage}{\textwidth}
\begin{tabularx}{\textwidth}{|l|X|l|X|}
\hline
\textbf{Course Title}       & Computer Networks  & \textbf{Course Code}       &   1914030 \\ \hline
\textbf{Credit Hours}       &  3 (Theoretical) & \textbf{ECTS Credit Hours}       &   6.9 \\ \hline
\end{tabularx}

\begin{tabularx}{\textwidth}{|l|X|}
\hline
\textbf{Pre-requisite}      & Operating Systems \\ \hline
\textbf{Co-requisite}       &  -- \\ \hline
\textbf{Text Book(s)}      & \begin{minipage}{.70\textwidth}
					\begin{itemize} \itemsep-0.4em
						\vspace{3mm}
						\item James F. Kurose and Keith W. Ross, Computer Networking: A Top-Down Approach, 5th Edition, Addison-Wesley Inc., 2009.
						\vspace{3mm}
					\end{itemize}
				\end{minipage}  \\ \hline
\textbf{Course Description} & \begin{minipage}{.70\textwidth}
					\vspace{3mm}
					Introduction to networks and digital communications with a focus on Internet protocols: Application layer architectures (client/server, peer-to-peer) and protocols (HTTP-web, SMTP-mail, etc), Transport layer operation: (reliable transport, congestion and flow control, UDP, TCP); Network layer operation - (routing, addressing, IPv4 and IPv6), Data Link layer operation (error detection/correction, access control, Ethernet, 802.11), Layer 2/3 protocols (MPLS); selected current topics such as: security, multimedia protocols, quality of Service, mobility, wireless networking, emerging protocols, network management.

					\vspace{3mm}
					\end{minipage} \\ \hline
\end{tabularx}
\end{minipage}


\bigskip
\bigskip



\begin{minipage}{\textwidth}
\begin{tabularx}{\textwidth}{|l|X|l|X|}
\hline
\textbf{Course Title}       &  Artificial Intelligence and Expert Systems & \textbf{Course Code}       &  1916028 \\ \hline
\textbf{Credit Hours}       &  3 (Theoretical) & \textbf{ECTS Credit Hours}       &   6.9 \\ \hline
\end{tabularx}

\begin{tabularx}{\textwidth}{|l|X|}
\hline
\textbf{Pre-requisite}      &  Data Structures \\ \hline
\textbf{Co-requisite}       &  -- \\ \hline
\textbf{Text Book(s)}      & \begin{minipage}{.70\textwidth}
					\begin{itemize} \itemsep-0.4em
						\vspace{3mm}
						\item S. Russel and P. Norving, Artificial Intelligence: A Modern Approach. 3rd Edition, Prentice Hall, 2010.
						\vspace{3mm}
					\end{itemize}
				\end{minipage}  \\ \hline
\textbf{Course Description} & \begin{minipage}{.70\textwidth}
					\vspace{3mm}
					The course deals with a broad range of artificial intelligence (AI) topics. It introduces the programming languages for artificial intelligence Prolog and Lisp. The course begins with an introduction to AI applications, predicate calculus, and state space search. Then it delves into some central areas of artificial intelligence such as heuristic strategies, problem solving, knowledge representation, expert systems, and machine learning. 

					\vspace{3mm}
					\end{minipage} \\ \hline
\end{tabularx}
\end{minipage}


\bigskip
\bigskip



\begin{minipage}{\textwidth}
\begin{tabularx}{\textwidth}{|l|X|l|X|}
\hline
\textbf{Course Title}       &   Fundamentals of Compiler Design & \textbf{Course Code}       &  1912012 \\ \hline
\textbf{Credit Hours}       &  3 (Theoretical) & \textbf{ECTS Credit Hours}       &   6.9 \\ \hline
\end{tabularx}

\begin{tabularx}{\textwidth}{|l|X|}
\hline
\textbf{Pre-requisite}      &   Data Structures \\ \hline
\textbf{Co-requisite}       &  -- \\ \hline
\textbf{Text Book(s)}      & \begin{minipage}{.70\textwidth}
					\begin{itemize} \itemsep-0.4em
						\vspace{3mm}
						\item Alfred V. Aho, Ravi Sethi, and Jeffrey D. Ullman, Compilers: Principles, Techniques, and Tools. Second Edition, Boston: Addison-Wesly, 2007.
						\vspace{3mm}
					\end{itemize}
				\end{minipage}  \\ \hline
\textbf{Course Description} & \begin{minipage}{.70\textwidth}
					\vspace{3mm}
					This course explores the principles, algorithms, and data structures involved in the design and
					construction of compilers. Topics include finite-state machines, lexical analysis, context-free
					grammars and other parsing techniques, symbol tables and an introduction to intermediate code
					generation.

					\vspace{3mm}
					\end{minipage} \\ \hline
\end{tabularx}
\end{minipage}


\bigskip
\bigskip


\begin{minipage}{\textwidth}
\begin{tabularx}{\textwidth}{|l|X|l|X|}
\hline
\textbf{Course Title}       &  Operating Systems Laboratory  & \textbf{Course Code}       & 1912024  \\ \hline
\textbf{Credit Hours}       &  1 (Practical) & \textbf{ECTS Credit Hours}       &   6.9 \\ \hline
\end{tabularx}

\begin{tabularx}{\textwidth}{|l|X|}
\hline
\textbf{Pre-requisite}      &  Operating Systems \\ \hline
\textbf{Co-requisite}       &  -- \\ \hline
\textbf{Text Book(s)}      & \begin{minipage}{.70\textwidth}
					\begin{itemize} \itemsep-0.4em
						\vspace{3mm}
						\item K. Wall, M. Watson, and M. wWhitis, Linux Programming Unleashed. Sams Publishers Inc., 1999.
						\item M. K. Dallheimer, T. Dawson, L. Kaufman, M. Welsh, Running Linux. O'Reilly, 2002.
						\vspace{3mm}
					\end{itemize}
				\end{minipage}  \\ \hline
\textbf{Course Description} & \begin{minipage}{.70\textwidth}
					\vspace{3mm}
					Testing Operating Systems subjects practically.

					\vspace{3mm}
					\end{minipage} \\ \hline
\end{tabularx}
\end{minipage}


\bigskip
\bigskip




\begin{minipage}{\textwidth}
\begin{tabularx}{\textwidth}{|l|X|l|X|}
\hline
\textbf{Course Title}       &  Microprocessor Laboratory & \textbf{Course Code}       &  1914011 \\ \hline
\textbf{Credit Hours}       &  1 (Practical) & \textbf{ECTS Credit Hours}       &   6.9 \\ \hline
\end{tabularx}

\begin{tabularx}{\textwidth}{|l|X|}
\hline
\textbf{Pre-requisite}      &  Microprocessors and Assembly Language \\ \hline
\textbf{Co-requisite}       &  -- \\ \hline
\textbf{Text Book(s)}      & \begin{minipage}{.70\textwidth}
					\begin{itemize} \itemsep-0.4em
						\vspace{3mm}
						\item John Uffenbeck, The 8086/8088 Family: Design, Programming, and Interfacing. 3rd Edition, Prentice Hall, 2001.
						\vspace{3mm}
					\end{itemize}
				\end{minipage}  \\ \hline
\textbf{Course Description} & \begin{minipage}{.70\textwidth}
					\vspace{3mm}
					This course is related to topics regarding to microprocessors and basic concepts of
					designing and developing them practically.

					\vspace{3mm}
					\end{minipage} \\ \hline
\end{tabularx}
\end{minipage}


\bigskip
\bigskip



\begin{minipage}{\textwidth}
\begin{tabularx}{\textwidth}{|l|X|l|X|}
\hline
\textbf{Course Title}       &  Computer Networks Laboratory & \textbf{Course Code}       &   1914018 \\ \hline
\textbf{Credit Hours}       &  1 (Practical) & \textbf{ECTS Credit Hours}       &   6.9 \\ \hline
\end{tabularx}

\begin{tabularx}{\textwidth}{|l|X|}
\hline
\textbf{Pre-requisite}      &  -- \\ \hline
\textbf{Co-requisite}       &  Computer Network \\ \hline
\textbf{Text Book(s)}      & \begin{minipage}{.70\textwidth}
					\begin{itemize} \itemsep-0.4em
						\vspace{3mm}

						\item S. Panwar, S. Mao, J. Ryoo, Y. Li, TCP/IP Essentials: A lab-Based Approach. Cambridge University Press, 2004.
						\vspace{3mm}
					\end{itemize}
				\end{minipage}  \\ \hline
\textbf{Course Description} & \begin{minipage}{.70\textwidth}
					\vspace{3mm}
					
					 Testing Computer Networks subjects practically
					\vspace{3mm}
					\end{minipage} \\ \hline
\end{tabularx}
\end{minipage}


\bigskip
\bigskip



\begin{minipage}{\textwidth}
\begin{tabularx}{\textwidth}{|l|X|l|X|}
\hline
\textbf{Course Title}       &  Systems Analysis and Design & \textbf{Course Code}       &   \\ \hline
\textbf{Credit Hours}       &  3 (Theoretical) & \textbf{ECTS Credit Hours}       &   6.9 \\ \hline
\end{tabularx}

\begin{tabularx}{\textwidth}{|l|X|}
\hline
\textbf{Pre-requisite}      &  Advanced Programming \\ \hline
\textbf{Co-requisite}       &  -- \\ \hline
\textbf{Text Book(s)}      & \begin{minipage}{.70\textwidth}
					\begin{itemize} \itemsep-0.4em
						\vspace{3mm}
						\item L. D. Bentley and J. L. Whitten, Systems Analysis and Design for the Global Enterprise. 7th Edition, McGraw-Hill, 2007.
						\item C. Larman, Applying UML and Patterns: An Introduction to Object-Oriented Analysis and Design and Iterative Development. Addison Wesley, 2004.
						\vspace{3mm}
					\end{itemize}
				\end{minipage}  \\ \hline
\textbf{Course Description} & \begin{minipage}{.70\textwidth}
					\vspace{3mm}
					System analysis and design deal with planning the development of information systems through understanding and specifying in detail what a system should do and how the components of the system should be implemented and work together. System analysts solve business problems through analyzing the requirements of information systems and designing such systems by applying analysis and design techniques. This course deals with the concepts, skills, methodologies, techniques, tools, and perspectives essential for systems analysts.

					\vspace{3mm}
					\end{minipage} \\ \hline
\end{tabularx}
\end{minipage}


\bigskip
\bigskip



\begin{minipage}{\textwidth}
\begin{tabularx}{\textwidth}{|l|X|l|X|}
\hline
\textbf{Course Title}       &  Principles of Database Design & \textbf{Course Code}       &  1912030 \\ \hline
\textbf{Credit Hours}       &  3 (Theoretical) & \textbf{ECTS Credit Hours}       &   6.9 \\ \hline
\end{tabularx}

\begin{tabularx}{\textwidth}{|l|X|}
\hline
\textbf{Pre-requisite}      &  Data Structures \\ \hline
\textbf{Co-requisite}       &  -- \\ \hline
\textbf{Text Book(s)}      & \begin{minipage}{.70\textwidth}
					\begin{itemize} \itemsep-0.4em
						\vspace{3mm}
						\item  A. Silberschatz, H. Korth, S. Sudarshan, Database System Concepts, 6th Edition, McGraw-Hill, 2009. 
						\item R. Ramakrishnan and J. Gehrke, Database Management Systems. 3rd Edition. McGraw-Hill Inc., 2003.
						\vspace{3mm}
					\end{itemize}
				\end{minipage}  \\ \hline
\textbf{Course Description} & \begin{minipage}{.70\textwidth}
					\vspace{3mm}
					The course aims to give a broad introduction to relational database systems,
					including the relational data model, query languages, index and file structures, query processing and
					optimization, concurrency and recovery, transaction management, and database design, plus optional
					material if time permits. 

					\vspace{3mm}
					\end{minipage} \\ \hline
\end{tabularx}
\end{minipage}


\bigskip
\bigskip



\begin{minipage}{\textwidth}
\begin{tabularx}{\textwidth}{|l|X|l|X|}
\hline
\textbf{Course Title}       &  Design of Programming Languages & \textbf{Course Code}       &   1912031 \\ \hline
\textbf{Credit Hours}       &  3 (Theoretical) & \textbf{ECTS Credit Hours}       &   6.9 \\ \hline
\end{tabularx}

\begin{tabularx}{\textwidth}{|l|X|}
\hline
\textbf{Pre-requisite}      &  Fundamentals of Compiler Design \\ \hline
\textbf{Co-requisite}       &  -- \\ \hline
\textbf{Text Book(s)}      & \begin{minipage}{.70\textwidth}
					\begin{itemize} \itemsep-0.4em
						\vspace{3mm}
						\item John Mitchell, Concepts in Programming Languages, Cambridge University Press, 2004.
						\item Robert W. Sebesta, Concepts of Programming Languages, 10th edition, Addison Wesley, 2012.
						\vspace{3mm}
					\end{itemize}
				\end{minipage}  \\ \hline
\textbf{Course Description} & \begin{minipage}{.70\textwidth}
					\vspace{3mm}
					Design features of modern programming languages, including flow control
					mechanisms and data structures; techniques for implementation of these features. Topics: Principles of programming languages, programming paradigms, and language trade-offs. Scope
					and bindings, data types, subprograms, semantics, syntax and its specification. Programming in
					representative languages.

					\vspace{3mm}
					\end{minipage} \\ \hline
\end{tabularx}
\end{minipage}


\bigskip
\bigskip


\begin{minipage}{\textwidth}
\begin{tabularx}{\textwidth}{|l|X|l|X|}
\hline
\textbf{Course Title}       &  Software Engineering  & \textbf{Course Code}       & 1912033  \\ \hline
\textbf{Credit Hours}       &  3 (Theoretical) & \textbf{ECTS Credit Hours}       &  6.9 \\ \hline
\end{tabularx}

\begin{tabularx}{\textwidth}{|l|X|}
\hline
\textbf{Pre-requisite}      &  Systems Analysis and Design \\ \hline
\textbf{Co-requisite}       &  -- \\ \hline
\textbf{Text Book(s)}      & \begin{minipage}{.70\textwidth}
					\begin{itemize} \itemsep-0.4em
						\vspace{3mm}
						\item Sommerville, Ian, Software Engineering, Fifth Edition. Addison-Wesley , 1996.
						\item Roger s. Pressman, Software Engineering: A Practitioner's Approach. McGraw Hill, 7th Edition, 2011.
						\vspace{3mm}
					\end{itemize}
				\end{minipage}  \\ \hline
\textbf{Course Description} & \begin{minipage}{.70\textwidth}
					\vspace{3mm}
					Introduction to software life cycle models.
					Software requirements engineering, formal specification and validation. Techniques for software
					design and testing. Cost estimation models. Issues in software quality assurance and software
					maintenance.

					\vspace{3mm}
					\end{minipage} \\ \hline
\end{tabularx}
\end{minipage}


\bigskip
\bigskip

\begin{minipage}{\textwidth}
\begin{tabularx}{\textwidth}{|l|X|l|X|}
\hline
\textbf{Course Title}       &   Internet Engineering & \textbf{Course Code}       &   1912016\\ \hline
\textbf{Credit Hours}       &   3 (Theoretical) & \textbf{ECTS Credit Hours}       & 6.9  \\ \hline
\end{tabularx}

\begin{tabularx}{\textwidth}{|l|X|}
\hline
\textbf{Pre-requisite}      &  Computer Networks \\ \hline
\textbf{Co-requisite}       &  Principles of Database Design \\ \hline
\textbf{Text Book(s)}      & \begin{minipage}{.70\textwidth}
					\begin{itemize} \itemsep-0.4em
						\vspace{3mm}
						\item M. Fowler, Patterns of Enterprise Application Architecture. Addison-Wesly, 2003.
						\vspace{3mm}
					\end{itemize}
				\end{minipage}  \\ \hline
\textbf{Course Description} & \begin{minipage}{.70\textwidth}
					\vspace{3mm}

					This course is an introduction to programming for the World Wide Web. We’ll learn about the relationship
					between clients and servers, how web pages are constructed, and how the internet works. We’ll examine several
					technologies in depth: 
					\newline
					\begin{enumerate}
					  \item  HyperText Markup Language (HTML) for authoring web pages
					  \item Cascading Style Sheets (CSS) for applying stylistic information to web pages
					  \item  JavaScript for creating interactive web pages
					  \item Asynchronous JavaScript and XML (Ajax) for enhanced web interaction and applications.
					  \item  PHP Hypertext Processor for generating dynamic pages on a web server
					  \item   Structure Query Language (SQL) for interacting with databases
					\end{enumerate}

					\vspace{3mm}
				\end{minipage} \\ \hline
\end{tabularx}
\end{minipage}


\bigskip
\bigskip

\begin{minipage}{\textwidth}
\begin{tabularx}{\textwidth}{|l|X|l|X|}
\hline
\textbf{Course Title}       &  Foundations of Computer Vision & \textbf{Course Code}       &  1916033 \\ \hline
\textbf{Credit Hours}       &  3 (Theoretical) & \textbf{ECTS Credit Hours}       &  6.9 \\ \hline
\end{tabularx}

\begin{tabularx}{\textwidth}{|l|X|}
\hline
\textbf{Pre-requisite}      &   Principles of Computational Intelligence \\ \hline
\textbf{Co-requisite}       &  -- \\ \hline
\textbf{Text Book(s)}      & \begin{minipage}{.70\textwidth}
					\begin{itemize} \itemsep-0.4em
						\vspace{3mm}
						\item R. C. Gonzalez and R. E. Woods, Digital Image Processing. 3rd Edition, Prentice-Hall, 2008.
						\item R. Jain, R. Kasturi, B. G. Schunck, Machine Vision. McGraw-Hill, 1995.
						\vspace{3mm}
					\end{itemize}
				\end{minipage}  \\ \hline
\textbf{Course Description} & \begin{minipage}{.70\textwidth}
					\vspace{3mm}
					his course provides an introduction to computer vision, including fundamentals of image formation, camera imaging geometry, feature detection and matching, stereo, motion estimation and tracking, image classification, scene understanding, and deep learning with neural networks. We will develop basic methods for applications that include finding known models in images, depth recovery from stereo, camera calibration, image stabilization, automated alignment, tracking, boundary detection, and recognition. We will develop the intuitions and mathematics of the methods in class, and then learn about the difference between theory and practice in projects.

					\vspace{3mm}
					\end{minipage} \\ \hline
\end{tabularx}
\end{minipage}


\bigskip
\bigskip


\begin{minipage}{\textwidth}
\begin{tabularx}{\textwidth}{|l|X|l|X|}
\hline
\textbf{Course Title}       &   Foundations of Speech and Language Processing & \textbf{Course Code}       & 1916032  \\ \hline
\textbf{Credit Hours}       &   3 (Theoretical) & \textbf{ECTS Credit Hours}       & 6.9  \\ \hline
\end{tabularx}

\begin{tabularx}{\textwidth}{|l|X|}
\hline
\textbf{Pre-requisite}      &   Signal and Systems, Engineering Probablitiy and Statistics \\ \hline
\textbf{Co-requisite}       &  -- \\ \hline
\textbf{Text Book(s)}      & \begin{minipage}{.70\textwidth}
					\begin{itemize} \itemsep-0.4em
						\vspace{3mm}
						\item Jurafsky and Martin, SPEECH and LANGUAGE PROCESSING: An Introduction to Natural Language Processing, Computational Linguistics, and Speech Recognition, Second Edition, McGraw Hill, 2008.
						\item Lawrence Rabiner and Biing-Hwang Juang. Fundamentals of Speech Recognition. Prentice Hall, 1993.
						\vspace{3mm}
					\end{itemize}
				\end{minipage}  \\ \hline
\textbf{Course Description} & \begin{minipage}{.70\textwidth}
					\vspace{3mm}
					Fundamentals of natural language processing, automatic speech recognition and speech synthesis; lab projects concentrating on building systems to process written and/or spoken language. 

					\vspace{3mm}
					\end{minipage} \\ \hline
\end{tabularx}
\end{minipage}


\bigskip
\bigskip


\begin{minipage}{\textwidth}
\begin{tabularx}{\textwidth}{|l|X|l|X|}
\hline
\textbf{Course Title}       &   Engineering Economy & \textbf{Course Code}       &  1112028 \\ \hline
\textbf{Credit Hours}       &  3 (Theoretical) & \textbf{ECTS Credit Hours}       &  6.9 \\ \hline
\end{tabularx}

\begin{tabularx}{\textwidth}{|l|X|}
\hline
\textbf{Pre-requisite}      &  -- \\ \hline
\textbf{Co-requisite}       &  -- \\ \hline
\textbf{Text Book(s)}      & \begin{minipage}{.70\textwidth}
					\begin{itemize} \itemsep-0.4em
						\vspace{3mm}
						\item William G. Sullivan, Elin M. Wicks, C. Patric Koelling, Engineering Economy, 15th Edition, Prentice Hall, 2011.
						\item L. T. A. Blank, A. J. Tarquin, Engineering Economy. 6th Edition, McGraw-Hill, New York, 2005.
						\vspace{3mm}
					\end{itemize}
				\end{minipage}  \\ \hline
\textbf{Course Description} & \begin{minipage}{.70\textwidth}
					\vspace{3mm}
					Engineering Economy is the process of making rational and intelligent decisions associated with the allocation of
					scarce resources in circumstances in which alternatives can be enumerated. This course provides engineers with
					skills to assess the costs and benefits of engineering investments, such as product and technology development
					programs and capital purchases. It also presents the framework for selecting among alternative designs, for
					managing technologies over their lifecycles, and for evaluating the finances of new ventures/projects.

					\vspace{3mm}
					\end{minipage} \\ \hline
\end{tabularx}
\end{minipage}


\bigskip
\bigskip


\begin{minipage}{\textwidth}
\begin{tabularx}{\textwidth}{|l|X|l|X|}
\hline
\textbf{Course Title}       &   Software Testing & \textbf{Course Code}       &  1912040  \\ \hline
\textbf{Credit Hours}       &  3 (Theoretical) & \textbf{ECTS Credit Hours}       &   6.9 \\ \hline
\end{tabularx}

\begin{tabularx}{\textwidth}{|l|X|}
\hline
\textbf{Pre-requisite}      &  Systems Analysis and Design \\ \hline
\textbf{Co-requisite}       &  -- \\ \hline
\textbf{Text Book(s)}      & \begin{minipage}{.70\textwidth}
					\begin{itemize} \itemsep-0.4em
						\vspace{3mm}
						\item P. Ammann, J. Offutt, Introduction to Software Testing. Cambridge University Press, 2008.
						\vspace{3mm}
					\end{itemize}
				\end{minipage}  \\ \hline
\textbf{Course Description} & \begin{minipage}{.70\textwidth}
					\vspace{3mm}
					This course has two closely related themes. First, more than half the effort in software development is devoted to activities related to testing, including test design, execution and evaluation. In this course, you will learn quantitative, technical, practical methods that software engineers and developers can use to test their software, both during and at the end of development. Second, more than half of software development effort is not new development, but maintenance activities such as adding new features, correcting problems, migrating to new platforms, and integrating third-party components into new projects. These two themes are intertwined because much of the effort during maintenance is testing the changes, and much of the effort in testing is about evaluating changes. 

					This course covers these two themes quantitatively, with a solid basis in theory and with practical applications. These topics will be useful to strong programmers in the Computer Science program, as well as engineers, physical scientists, and mathematicians who regularly integrate software components as part of their work. The topic of this course is of interest to and accessible to students in a wide variety of specializations. 

					\vspace{3mm}
					\end{minipage} \\ \hline
\end{tabularx}
\end{minipage}


\bigskip
\bigskip


\begin{minipage}{\textwidth}
\begin{tabularx}{\textwidth}{|l|X|l|X|}
\hline
\textbf{Course Title}       &   Fundamentals of Robotics & \textbf{Course Code}       &  1916034 \\ \hline
\textbf{Credit Hours}       &  3 (Theoretical) & \textbf{ECTS Credit Hours}       &   6.9 \\ \hline
\end{tabularx}

\begin{tabularx}{\textwidth}{|l|X|}
\hline
\textbf{Pre-requisite}      &  Signal and Systems \\ \hline
\textbf{Co-requisite}       &  -- \\ \hline
\textbf{Text Book(s)}      & \begin{minipage}{.70\textwidth}
					\begin{itemize} \itemsep-0.4em
						\vspace{3mm}
						\item John J. Craig, Introduction to Robotics: Mechanics and Control. 3rd Edition, Prentice Hall, 2004.
						\vspace{3mm}
					\end{itemize}
				\end{minipage}  \\ \hline
\textbf{Course Description} & \begin{minipage}{.70\textwidth}
					\vspace{3mm}
					Robotics as an application draws from many different fields and allows automation of products as diverse as cars, vacuum cleaners, and factories. This course is a challenging introduction to basic computational concepts used broadly in robotics. Topics include simulation, kinematics, control, optimization, and probabilistic inference. The mathematical basis of each area is emphasized, and concepts are motivated using common robotics applications and programming exercises.

					\vspace{3mm}
					\end{minipage} \\ \hline
\end{tabularx}
\end{minipage}


\bigskip
\bigskip

\begin{minipage}{\textwidth}
\begin{tabularx}{\textwidth}{|l|X|l|X|}
\hline
\textbf{Course Title}       &  Principles of Computational Intelligence & \textbf{Course Code}       &  1916031 \\ \hline
\textbf{Credit Hours}       &   3 (Theoretical) & \textbf{ECTS Credit Hours}       &  6.9 \\ \hline
\end{tabularx}

\begin{tabularx}{\textwidth}{|l|X|}
\hline
\textbf{Pre-requisite}      &  Advanced Programming \\ \hline
\textbf{Co-requisite}       &  -- \\ \hline
\textbf{Text Book(s)}      & \begin{minipage}{.70\textwidth}
					\begin{itemize} \itemsep-0.4em
						\vspace{3mm}
						\item T. J. Ross, Fuzzy Logic with Engineering Applications. John Wily and Sons, 2004.
						\item David B. Fogel, Thomas Back, and Zbingniew Michalewicz, Evolutionary Computation: Basic algorithms and operators, Institute of Physics Publishing, 2000.
						\vspace{3mm}
					\end{itemize}
				\end{minipage}  \\ \hline
\textbf{Course Description} & \begin{minipage}{.70\textwidth}
					\vspace{3mm}
					Computational Intelligence covers a wide range of issues that developed in parallel with, or in competition to, symbolic AI. The major constituents of the field are bio-inspired computing – which deals with an ever expanding number of biologically related techniques – and fuzzy logic – which deals with reasoning under conditions of vagueness. In this course we will explore a number of topics that are core to Computational Intelligence (e.g. neural nets and evolutionary computing) and these will lead into some state-of-the-art approaches (such as fuzzy model-based reasoning and learning).

					\vspace{3mm}
					\end{minipage} \\ \hline
\end{tabularx}
\end{minipage}


\bigskip
\bigskip




\end{document}



