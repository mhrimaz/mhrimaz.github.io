% LaTeX Curriculum Vitae Template
%
% Copyright (C) 2004-2009 Jason Blevins <jrblevin@sdf.lonestar.org>
% http://jblevins.org/projects/cv-template/
%
% You may use use this document as a template to create your own CV
% and you may redistribute the source code freely. No attribution is
% required in any resulting documents. I do ask that you please leave
% this notice and the above URL in the source code if you choose to
% redistribute this file.

\documentclass[letterpaper]{article}

\usepackage{hyperref}
\usepackage{geometry}
\usepackage{tabulary}
\usepackage{fontawesome}
% Comment the following lines to use the default Computer Modern font
% instead of the Palatino font provided by the mathpazo package.
% Remove the 'osf' bit if you don't like the old style figures.
\usepackage[T1]{fontenc}
\usepackage[sc,osf]{mathpazo}

% Set your name here
\def\name{Mohammad Hossein Rimaz}

% Replace this with a link to your CV if you like, or set it empty
% (as in \def\footerlink{}) to remove the link in the footer:
\def\footerlink{http://mhrimaz.com}

% The following metadata will show up in the PDF properties
\hypersetup{
  colorlinks = true,
  urlcolor = black,
  pdfauthor = {\name},
  pdfkeywords = {rimaz, resume, cv},
  pdftitle = {\name: Mohammad Hossein Rimaz Curriculum Vitae},
  pdfsubject = {Mohammad Hossein Rimaz Curriculum Vitae},
  pdfpagemode = UseNone
}

\geometry{
  bottom=0.65in,
  left=0.65in,
  right=0.65in,
  top=0.90in
}

% Customize page headers
\pagestyle{myheadings}
\markright{\name}
\thispagestyle{empty}

% Custom section fonts
\usepackage{sectsty}
\sectionfont{\rmfamily\mdseries\Large}
\subsectionfont{\rmfamily\mdseries\itshape\large}

% Other possible font commands include:
% \ttfamily for teletype,
% \sffamily for sans serif,
% \bfseries for bold,
% \scshape for small caps,
% \normalsize, \large, \Large, \LARGE sizes.

% Don't indent paragraphs.
\setlength\parindent{0em}

% Make lists without bullets
\renewenvironment{itemize}{
  \begin{list}{}{
    \setlength{\leftmargin}{1.5em}
  }
}{
  \end{list}
}

\begin{document}

% Place name at left
{\huge \bf \name}

% Alternatively, print name centered and bold:
%\centerline{\huge \bf \name}

\vspace{0.25in}

\begin{minipage}{0.45\linewidth}
  Tavanir St., Vali-Asr St. \\
  Tehran, Tehran $1434885485$, Iran
\end{minipage}
\begin{minipage}{0.45\linewidth}
  \begin{tabular}{ll}
  \faEnvelope  \thinspace Email: & \href{mailto:mhrimaz@acm.org}{\tt mhrimaz@acm.org} \\
   \faPhone \thinspace Phone: & $+98~933~888~0256$ \\
    \faHome \thinspace Homepage: & \href{https://mhrimaz.com}{\tt https://mhrimaz.com} \\
    \faGithub \thinspace GitHub: & \href{https://github.com/mhrimaz}{\tt https://github.com/mhrimaz} 
  \end{tabular}
\end{minipage}


\section*{Education}

{\renewcommand{\arraystretch}{1.2}
\begin{tabular}{l l}
2014--2019 & BSc, Software Engineering, K. N. Toosi University of Technology,\\
 & Cumulative GPA: 15.85/20 (US CGPA: 3.14/4, ECTS: 2.2/4) (top 15\% in the class) \\
 & \textbf{Selected Courses}: Java, Software Testing, Software Engineering. (all grades 20/20)\\& Algorithm Design (18.5/20), Artificial Intelligence \& Expert Systems (19.3/20)
\end{tabular}
}

\section*{Fields of Interests}

\begin{itemize}
	\item Machine Learning, Data Mining, Information Retrieval, Recommender Systems
\end{itemize}



\section*{Honors and Awards}

{\renewcommand{\arraystretch}{1.2}
\begin{tabular}{l l}
April 2018 & \textbf{3rd place} in RoboCup Iran Open 2018 International Competitions, UAV Indoor League, Team KN2C.
\end{tabular}

\section*{Publications}

\begin{itemize}

\item \textbf{Mohammad Hossein Rimaz}, Mehdi Elahi, Farshad Bakhshandegan Moghadam, Christoph Trattner, Reza Hosseini, and Marko Tkalčič. $2019$. Exploring the Power of Visual Features for Recommendation of Movies. In {\it 27th Conference on User Modeling, Adaptation and Personalization} {(UMAP ’19)}, June 9-12, 2019, Larnaca, Cyprus. ACM, New York, NY, USA, 6 pages. \href{https://doi.org/10.1145/3320435.3320470}{\tt https://doi.org/10.1145/3320435.3320470}
\end{itemize}


\section*{Projects}

{\renewcommand{\arraystretch}{1.2}
\begin{tabular}{l p{15cm}}
Summer 2019  & \textbf{Hi-Rec} \\ & Role: Researcher \& Developer | Supervisor: \href{https://scholar.google.com/citations?user=aUWF7LYAAAAJ&hl=en}{\tt M. Elahi} \\&Hi-Rec is a Cross-Platform, Open Source, Extensible and Easy to Use Java framework for recommender systems. Responsibilites: Redesigning the GUI architecture, refactoring the core engine, implementing algorithms.
\\
Summer 2017 & \textbf{Autonomous landing on Artificial Landmark} \\ & Role: Developer in KN2C Robotic Lab, Aerial Unmanned Vehicle Team \\&Designing artificial landmarks and developing real-time vision-based autonomous landing algorithm (outdoor and indoor environments). Languages: OpenCV and C++.
\\
Jan – April 2017 & \textbf{Cloud Billing System} \\& Role: Researcher, Developer | Supervisors: \href{http://wp.kntu.ac.ir/ahmadi/}{\tt A. Ahmadi} and \href{http://wp.kntu.ac.ir/sedighian/}{\tt  S. Kashi}.
\\&
Investigating open-source billing systems. Examining time series databases.
\\
Summer 2016 &  \textbf{ODE and PDE MathTools} \\&
Role: Developer, Web Site Designer | Supervisor: \href{http://wp.kntu.ac.ir/aliakbarian/}{\tt H. Aliakbarian} \\&
Several mathematical visualizations in JavaFX.

\end{tabular}

\section*{Language}

Farsi (Native), English (Professional Proficiency), Deutsch (A2)\newline



\begin{tabular}{|c|c|}
\textbf{TOEFL: 106} & \textbf{GRE} \\
Reading:28, Listening:27, Speaking:24, Writing:27. & Verbal:145 (27\%), Quantitative :164(86\%), Writing: 3.5 (41\%).\\
\end{tabular}

\section*{Technical Skills}

{\renewcommand{\arraystretch}{1.2}
\begin{tabular}{l l}
Languages & Java, Python, C++, Scala.
\\
Programming & Object Oriented and Functional Programming, OOP Design Patterns.
\\
Concepts & Concurrency and Parallelism such as Java Fork/Join, Akka Actor Model, Async programming, \\ & Version Control Systems such as Git and GitHub,REST and SOAP Web Services.
\\
IDEs & IntelliJ IDEA, Apache NetBeans, Eclipse.
\\
Databases & Oracle 12c, MongoDB, SQLite.
\\
Libraries & Akka, Git, Hibernate, JPA, JUnit, Mockito, JavaFX, Maven, JDBC, scikit, numpy.
\end{tabular}


\section*{Teaching Experience}

{\renewcommand{\arraystretch}{1.2}
\begin{tabular}{l l}
Fall 2015--Spring 2019 & Teaching assistant, \textbf{Advanced Programming with Java}, KNTU, \\&Instructors: \href{https://scholar.google.com/citations?hl=en&user=kf0UQKMAAAAJ}{\tt Mehdi Esnaashari}, \href{http://wp.kntu.ac.ir/zamanian/}{\tt Mahdi Zamanian}, \href{http://wwwlgis.informatik.uni-kl.de/cms/dbis/staff/formermembers/izadi/}{\tt Sayyed Kamyar Izadi} 
\\
Spring 2017 & Teaching assistant, \textbf{Algorithms}, KNTU, Instructor: \href{https://scholar.google.com/citations?user=mGF6p48AAAAJ&hl=en}{\tt Amin Nikanjam} 
\\
Fall 2016 & Teaching assistant, \textbf{Data Structure}, KNTU, Instructor: \href{http://wp.kntu.ac.ir/bnasersharif/}{\tt Babak Nasersharif} 

\end{tabular}


\section*{Community Outreach}

{\renewcommand{\arraystretch}{1.2}
\begin{tabular}{l l}
May 2018-July 2019 & Chairman of ACM Student Chapter, KNTU. Chapter Website: \href{https://kntu.acm.org}{\tt https://kntu.acm.org} 
\\
Sep 2017-July 2019 & Founder and leader of KNTU Java User Group (JUG). \href{https://www.youtube.com/channel/UCIZWx5W3uNAKI4ggooa7qNA}{\tt KNTU JUG YouTube Channel} 
\\
November 2016-2017 & Chairman of Computer Engineering Student's Scientific Chapter, KNTU.
\end{tabular}

\section*{Professional Membership}

\begin{itemize}
\item Association for Computing Machinery (2017-present). Membership Number: 7348433
\end{itemize}



\section*{Certificates}

{\renewcommand{\arraystretch}{1.2}
\begin{tabular}{l l}
\textbf{Coursera} & \textbf{Verified certificates from Coursera online MOOC platform.}
\\
\\
July 2018 &  Parallel, Concurrent, and Distributed Programming in Java Specialization |  \href{https://www.coursera.org/account/accomplishments/specialization/7XPLLTAAT4TG}{\tt Rice University}.
\\
September 2017 &  Intro to RecSys: Matrix Factorization and Advanced Techniques |  \href{https://www.coursera.org/account/accomplishments/certificate/6T2FRVVK76DY}{\tt University of Minnesota}.
\\
August 2017 &  Intro to RecSys: Evaluation and Metrics |  \href{https://www.coursera.org/account/accomplishments/certificate/6CVCA6AY7ESY}{\tt University of Minnesota}.
\\
March 2017 &  Intro to RecSys: Nearest Neighbor Collaborative Filtering                                               |  \href{https://www.coursera.org/account/accomplishments/verify/PZNSF9HR2TS4}{\tt University of Minnesota}.
\\
December 2016 & Intro to RecSys: Non-Personalized and Content-Based  |  \href{https://www.coursera.org/account/accomplishments/verify/DUQNZMH9CU7W}{\tt University of Minnesota}.
\\
November 2016 & Functional Program Design in Scala |  \href{https://www.coursera.org/account/accomplishments/certificate/LNSNRHKK4K44}{\tt École Polytechnique Fédérale de Lausanne}.
\\
September 2016 & Machine Learning                                                                                                  |  \href{https://www.coursera.org/account/accomplishments/certificate/JE788VNJ6TKS}{\tt Stanford University}.
\\
August 2016 & Algorithms on Strings |  \href{https://www.coursera.org/account/accomplishments/certificate/ZALUCS3TDQTD}{\tt University of California, San Diego}.
\\

July 2016 & Functional Programming Principles in Scala               |  \href{https://www.coursera.org/account/accomplishments/certificate/56M9Y2XYRHNL}{\tt École Polytechnique Fédérale de Lausanne}.
\\
July 2016 &  Algorithms on Graphs                  |  \href{https://www.coursera.org/account/accomplishments/certificate/D2GHQ9U8UFN2}{\tt University of California, San Diego}.
\\
March 2016 &  Advanced Data Structures in Java                                              |  \href{https://www.coursera.org/account/accomplishments/certificate/3UZ6VLS38VLW}{\tt University of California, San Diego}.

\end{tabular}





\end{document}
